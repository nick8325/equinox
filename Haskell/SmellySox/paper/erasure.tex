
\section{Sort Erasure}

Note: The idea of this section is to show different ways of removing the
sorts, discuss their advantages and disadvantages, and lastly
how some of the disadvantages can be remedied by monotonicity inference.


Remove sorts in example from introduction. Show that doesn't work.

%Show roughly how this works.

%Say that it is not always necessary.

%Explain the need for a calculus to show when it is necessary and when 
%it isn't.

%Introduce Monotonicity for FOL. Refer to Jasmin's paper.

%Say that we have implemented this in a tool called Monopoly and 
%that we will explain more about this in sections blah, blah and blah.


%One obvious approach is to simply erase all the sorts: that is,
%whenever we have a sorted quantification $(\forall x:S)\ldots$ or
%$(\exists x:S)\ldots$, we just turn it into an unsorted quantification
%$(\forall x)\ldots$ or $(\exists x)\ldots$.

%Unfortunately, this doesn't work. Consider the formula
%\begin{align*}
%& (\forall x y:A)(x = y) \\
%\land & (\exists x y:B)(x \neq y)
%\end{align*}
%This formula is satisfiable: for a model, let the domain of $A$ have
%size 1 and the domain of $B$ have size greater than 1. If we erase the
%sorts, though, we get $(\forall x y)(x = y) \land (\exists x y)(x \neq
%y)$, which is of the form $P \land \neg P$ and therefore
%unsatisfiable.

What went wrong? The problem is that the formula is satisfiable but only
when the domains of $A$ and $B$ have different sizes. When we erase
the types we flatten the two domains into one blah blah blah.

Blah blah the property we have: $erase(\formula)$ is satisfiable iff there is
a model of $\formula$ where all the domains have the same size.
Therefore, for erasure to be correct, we need to have this property:
$\formula$ is satisfiable iff it has a model where all the domains
have the same size.

\subsection{Introducing sorting predicates}

... For each sort $S$, we introduce a predicate $p_S$, of arity one,
so that $p_S(x)$ means ``$x$ has sort $S$''. A
$\forall$-quantification $(\forall x:S)\ldots$ becomes $(\forall
x)(p_S(x) \Rightarrow \ldots)$ and an $\exists$-quantification
$(\exists x:S)\ldots$ becomes $(\exists x)(p_S(x) \land \ldots)$. For
each function symbol we generate an axiom giving its sort; for
example, $f : (S \times T \Rightarrow U)$ induces an axiom
\begin{displaymath}
(\forall x y)(p_S(x) \land p_T(y) \Rightarrow p_U(f(x, y)))
\end{displaymath}

...

This translation is sound, but inefficient: we give the theorem prover
a lot of extra work to do in proving that things are of the right
sort. Whenever we instantiate an axiom of the form $(\forall
x:S)P(x)$ to get $P(t)$ we get an extra proof obligation $p_S(t)$.

\subsection{Introducing sorting functions}

Another approach is to for each sort S introduce a function $f_S$ of
arity one, so that for all $X$, $f_S(X)$ is equal to a constant of sort $S$. 

Any variable of type $S$ is then translated to $f_S(X)$.

...

\subsection{Introducing stuff only when necessary}

It is not always necessary to use a sorting predicate or function.
Example..

...

only necessary when the formula is not monotone. Example.
Next section blah.

When is it necessary to introduce sorting predicates?

Introducing predicates: satisfiability-preserving but makes the
formula monotone.
