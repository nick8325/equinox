\documentclass{llncs}

\usepackage{amsmath}
\usepackage{algorithm}
\usepackage{algpseudocode} 

\title{Sort it out with Monotonicity}
\author{Ann Lilliestr\"om and Nick Smallbone}
\institute{ Chalmers University of Technology, Gothenburg, Sweden
         \\ \email{\{annl,nicsma\}@chalmers.se}
          }

\begin{document}

\maketitle

\begin{abstract}
None of the famous theorem provers for first-order logic deal with
many-sorted problems. Ideally, we would like our provers to understand
many-sorted logic, but in the meantime, we want to be able to
\emph{translate} a sorted problem into an unsorted one so that the
usual crowd of theorem provers and model finders can solve it.

There are two obvious ways to do this. One is a) easy, b) efficient
and c) often absolutely wrong, while the other is a) always correct
and b) extremely wasteful. We present a technique called
\emph{monotonicity inference} that allows us to use a mixture of the
two techniques: the efficient one when sound and the wasteful one when
necessary. To demonstrate this technique we wrote a new tool,
Monopoly, the first of its kind as far as we know.
\end{abstract}

\section{Introduction}

Many problems are naturally expressed in sorted logic.

For example, in geometry, the universe can be divided into 
objects that are points, planes, lines etc...

sorted FOL has the same properties as  FOL.

Theorem provers - none exist for sorted logic...
Translate sorted to unsorted! Now we can use any FOL prover.

Sorted logic example:

\paragraph{Werewolves and villagers} As a running example, we model a
small part of the game of Werewolf. A village is infested with
werewolves, who are intent on killing the entire village. 
During the day, the werewolves pretend to be normal villagers but each
night, they rise up and murder one villager. Each day, the villagers
nominate one person that believe to be a werewolf and lynch them. If
all the werewolves are killed, the village wins; if all the normal
villagers are killed, the werewolves win.

\begin{example}
\label{ex:werewolf1}
\begin{eqnarray*}
\slain & \in & \villager \to o \\ 
\alterEgo & \in & \werewolf \to \villager \\
\victim & \in & \villager 
\end{eqnarray*}
\begin{align*}
\forall X,Y \in \werewolf \ & ( \alterEgo (X) = \alterEgo (Y) \Longrightarrow  X = Y) \\
 \forall X \in \villager\ & (\slain(X) \Longrightarrow X = \victim) \\
 & \slain(\victim) \\
 \forall X \in \werewolf \  & (\alterEgo(X) = \victim)
\end{align*} 
\end{example}

We want to use a model finder on this example to see if it is possible
for the villagers to win the game. But there are no model-finders for
sorted first-order logic!

What we can do though is to translate sorted logic to unsorted logic. 
(this is possible since sorted logic and unsorted logic are equally expressive) 
We can then use an existing model finder (or theorem prover) for unsorted logic.

How to translate?

To simply remove the sorts is unsound, as we shall see in section ...
Introduce sorting predicates - inefficient.

Use monotonicity to infer when it is necessary to introduce sorting predicates
and when it is safe to remove the sorts. (Mention Jasmins paper)

- were going to show you lots of cool stuff in the next sections. blahblah.



%TODO cite Enderton 1972 - translation sorted - unsorted by predicates.












\section{Sort Erasure}

One obvious approach is to simply erase all the sorts: that is,
whenever we have a sorted quantification $(\forall x:S)\ldots$ or
$(\exists x:S)\ldots$, we just turn it into an unsorted quantification
$(\forall x)\ldots$ or $(\exists x)\ldots$.

Unfortunately, this doesn't work. Consider the formula
\begin{align*}
& (\forall x y:A)(x = y) \\
\land & (\exists x y:B)(x \neq y)
\end{align*}
This formula is satisfiable: for a model, let the domain of $A$ have
size 1 and the domain of $B$ have size greater than 1. If we erase the
sorts, though, we get $(\forall x y)(x = y) \land (\exists x y)(x \neq
y)$, which is of the form $P \land \neg P$ and therefore
unsatisfiable.

\subsection{Introducing sorting predicates}

... For each sort $S$, we introduce a predicate $p_S$, of arity one,
so that $p_S(x)$ means ``$x$ has sort $S$''. A
$\forall$-quantification $(\forall x:S)\ldots$ becomes $(\forall
x)(p_S(x) \Rightarrow \ldots)$ and an $\exists$-quantification
$(\exists x:S)\ldots$ becomes $(\exists x)(p_S(x) \land \ldots)$. For
each function symbol we generate an axiom giving its sort; for
example, $f : (S \times T \Rightarrow U)$ induces an axiom
\begin{displaymath}
(\forall x y)(p_S(x) \land p_T(y) \Rightarrow p_U(f(x, y)))
\end{displaymath}

...

This translation is sound, but inefficient: we give the theorem prover
a lot of extra work to do in proving that things are of the right
sort. Whenever we instantiate an axiom of the form $(\forall
x:S)P(x)$ to get $P(t)$ we get an extra proof obligation $p_S(t)$.

\subsection{Introducing sorting functions}

Another approach is to for each sort S introduce a function $f_S$ of
arity one, so that for all $X$, $f_S(X)$ is equal to a constant of sort $S$. 

Any variable of type $S$ is then translated to $f_S(X)$.

...

\subsection{Introducing stuff only when necessary}

...

\section{Monotonicity calculus for first-order logic}

\label{sec_monotonicity}

In this section we treat monotonicity more formally and give a
calculus for inferring monotonicity of a formula. FIXME this is
rubbish

To begin with, we only consider monotonicity for unsorted formulae.
There is no real complication in the many-sorted case but the notation
is more cumbersome. Monotonicity is a semantic property rather than a
syntactic property of the formula.
\begin{definition}
An unsorted formula $\formula$ is \emph{monotone} if, for all $d$,
whenever $\formula$ is satisfiable over a domain of $d$ elements,
then $\formula$ is also satisfiable over a domain of $d+1$ elements.
\end{definition}

This definition only considers the formula's behaviour on
\emph{finite} domains---if $d$ is infinite, then $d+1 = d$ and the
condition is trivially satisfied. A trivial consequence is that if a
monotone formula is satisfiable for a domain of size $d$ then it is
satisfiable for all bigger domains.

FIXME write about satisfiability at a domain instead.

Several common classes of formulae are monotone:
\begin{itemize}
\item Any unsatisfiable formula is monotone because it has no models.
\item Any valid formula is monotone because it has a model no matter
  what the domain size.
\item A formula that only has infinite models is monotone because for
  all finite values of $d$ it has no models.
\item A formula that does not use $=$ is monotone, as we will see
  later.
\end{itemize}

The first two points suggest that monotonicity is related to
satisfiability and so should not be decidable. It is at least
semi-decidable: if a formula $\formula$ is not monotone then there
must be a domain size $d$ for which $\formula$ is satisfiable for
domains of size $d$ but not for domains of size $d+1$, and we can find
this $d$ using any model-finder. However, there is no algorithm that
can always tell us when a formula \emph{is} monotone; we leave the
proof to the appendix, but it is intuitively because monotonicity is
not easier than finite unsatisfiability.

What about a non-monotone formula? The simplest example is $(\forall
x, y)(x = y)$, which is satisfied if the domain contains a single
element but not if it contains two.

% A formula f is monotone in a type A, if

%   If f is satisfiable in a model where the domain of A has size n,
%   then it is also satisfiable for any domain size $>$ n.
%   (Unsatisfiable formulae are always monotone).

% A formula is anti-monotone in a type A if:

  % If f is unsatisfiable in a model where the domain of A has size n,
  % then it is also unsatisfiable for any domain size $>$ n.

Monotonicity inference is semi-decidable:

Reduce solving diophantine equations to monotonicity

To show that a problem is not monotone, we look for a domain size
k for which the problem is satisfiable, and not satisfiable for size k+1.
This can be done in finite time using a finite model finder.



In the many-sorted case we have one domain for each sort, which
complicates matters a little. We write $\M(\alpha)$ for the domain of
sort $\alpha$ in model $\M$.

\begin{definition}
A formula $\varphi$ is \emph{monotone} in a sort $\alpha$ if,
whenever we have a model $\M$ of $\varphi$ where $\M(\alpha)$ has $n$
elements, then there is a model $\M' \models \varphi$ where
$\M'(\alpha)$ has $n+1$ elements and for every sort $\beta \neq
\alpha$ we have $\M'(\beta) = \M(\beta)$.
\end{definition}

\subsection{Simple calculus (extend only by copy)}

- example where doesn't work

\subsection{Better calculus (extend by true/false/copy)}

\subsection{ "Biggest" sort, finding injective functions, infinite sorts...}

-Perhaps move to Future work.

\section{Monotonicity inference in practice with Monopoly}


We have implemented the monotonicity calculus as part of our tool
Monopoly.

Monopoly goes through the input formula once for each sort, checking if
the formula is monotone in that sort. If the answer is yes for some sort $\alpha$,
it is safe to let a quantification over $\alpha$ scope over the entire domain,
and the sort can thus be removed while preserving equisatisfiability. 

If the answer is no, a sorting predicate or sorting function (depending on the
flag provided by the user) must be introduced in order to make the sorted and the 
unsorted formula equisatisfiable.

\subsection{NP completeness of finding a model extension}

  To show that a formula is monotone in a sort $\alpha$, we must show that there is
  a consistent way of extending the domain of $\alpha$. 

  As explained in \ref{...}, if equality between two terms of type $\alpha$ occurs 
  positively in a clause, this may restrict the size of the domain of $\alpha$. But if 
  the equality is "guarded" by a predicate $p$ in the clause, then the clause is still monotone.

  For a predicate $p$ to be a guard, it must be extended by $false$ if it occurs negatively,
  and with $true$ if it occurs positively.To find if there is a consistent extension of $p$, 
  we must consider the entire formula, since different clauses may require conflicting 
  extensions for $p$, as in the following example:

  \begin{example}
\label{ex:extension_conflict}

%Consider the following theory $T1$.
\begin{eqnarray}
 & \p(X) \\
 & \neg \p(X) \lor X = \aconst 
\end{eqnarray}
%This theory is finitely unsatisfiable.
\end{example}
 
  The problem of finding such a model extension is NP complete, as a SAT-problem can be 
  encoded as a problem of finding a model extension as follows:

  Let each literal $l$ in the SAT-formula correspond to a predicate $p_l$ with one argument.
  The literal being $true$ in the SAT-problem corresponds to the predicate
  being extended by $true$, and likewise, if the literal is $false$ in the SAT-
  problem, the corresponding predicate should be extended by $false$. If the value of $l$ does
  not affect the satisfiability of the SAT-formula, the extension of the predicate does not
  matter either, so it can simply by extended by copying.

  In the SAT-problem, at least one literal in each clause must be $true$.
  We model this in the model extension problem, by for creating for each clause $(l_1 \vee ...\vee l_n)$ a formula
  $$ \neg p_{l_1}(X) \wedge ... \wedge \neg p_{l_n}(X) \Rightarrow X = c $$
  This formula is monotone in our calculus exactly when the variable $X$ 
  is guarded by at least one of the predicates $p_{l_1},...p_{l_n}$.
  Thus, if we can show that there is a consistent extension of the predicates
  which makes the formula monotone, then the original set of propositional 
  clauses is satisfiable.

  %TODO: Define "guardedness" - 

\subsection{...}

  In the implementation of Monopoly, we use a SAT-solver to find the context 
  in which to extend the predicates. It works as follows:

  For each predicate $p$ occuring in the sorted problem, we create two literals;
  $p_T$ and $p_F$. If $p_T$ is true, then $p$  is extended by true. 
  If $p_F$ is true, then $p$ is extended by false. Since a predicate can only
  be extended in one way, we add to our SAT formula the clause $\neg p_F \vee \neg p_T$.
  If both $p_T$ and $p_F$ are false, this means that $p$ is extended by copying.

  The SAT formula is the conjunction of SAT clauses and the constraint that no 
  predicates of the given sort may be true extended and false extended at the same time.

  The SAT clauses correspond to the clauses of the original formula.

  \begin{algorithm}[t]
\caption{Finding the Context}\label{alg:zoom} 
\begin{algorithmic}[1]
\Procedure{SATFormula}{$clauses$, $sort$} 
    \State \textbf{return}  

%   \State $n \gets size(theory) \;\mbox{div}\; 2$  
%   \While {$n \geq 1$}
%   \State partition $theory$ into partitions $p_i$ of size $n$ \Comment{one partition may have to be smaller than $n$}
%   \For {each partition $p_i$}
%      \If {\Call{TryFindFiniteModel}{$theory - p_i$} fails}               \Comment{try Paradox with given time-limit}
%         \State \textbf{return} \Call{Zoom}{$theory - p_i$}  \Comment{recursively find a better subset}
%      \EndIf 
%   \EndFor
%   \State $n \gets n \;\mbox{div}\; 2$                              \Comment{remove smaller parts in next iteration}
%   \EndWhile
%   \State \textbf{return} $theory$                   \Comment{every subtheory is finitely satisfiable}
\EndProcedure

\Procedure{SatClause}{$clause$, $sort$}
\EndProcedure
\end{algorithmic}
\end{algorithm}


  



\section{Results}

compare simple/better calculus.

compare with and without searching for "biggest" sort
with injective functions



\section{Future Work}

* Injective functions
* Sort inference
* Polymorphic predicates
* Improving the calculus (other types of predicate extensions)
* Other ways of representing sorts in unsorted FOL


\section{Conclusions}

We're awesome.
 


\end{document}
